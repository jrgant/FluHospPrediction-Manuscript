% Template for PLoS
% Version 3.5 March 2018
%
% % % % % % % % % % % % % % % % % % % % % %
%
% -- IMPORTANT NOTE
%
% This template contains comments intended
% to minimize problems and delays during our production
% process. Please follow the template instructions
% whenever possible.
%
% % % % % % % % % % % % % % % % % % % % % % %
%
% Once your paper is accepted for publication,
% PLEASE REMOVE ALL TRACKED CHANGES in this file
% and leave only the final text of your manuscript.
% PLOS recommends the use of latexdiff to track changes during review, as this will help to maintain a clean tex file.
% Visit https://www.ctan.org/pkg/latexdiff?lang=en for info or contact us at latex@plos.org.
%
%
% There are no restrictions on package use within the LaTeX files except that
% no packages listed in the template may be deleted.
%
% Please do not include colors or graphics in the text.
%
% The manuscript LaTeX source should be contained within a single file (do not use \input, \externaldocument, or similar commands).
%
% % % % % % % % % % % % % % % % % % % % % % %
%
% -- FIGURES AND TABLES
%
% Please include tables/figure captions directly after the paragraph where they are first cited in the text.
%
% DO NOT INCLUDE GRAPHICS IN YOUR MANUSCRIPT
% - Figures should be uploaded separately from your manuscript file.
% - Figures generated using LaTeX should be extracted and removed from the PDF before submission.
% - Figures containing multiple panels/subfigures must be combined into one image file before submission.
% For figure citations, please use "Fig" instead of "Figure".
% See http://journals.plos.org/plosone/s/figures for PLOS figure guidelines.
%
% Tables should be cell-based and may not contain:
% - spacing/line breaks within cells to alter layout or alignment
% - do not nest tabular environments (no tabular environments within tabular environments)
% - no graphics or colored text (cell background color/shading OK)
% See http://journals.plos.org/plosone/s/tables for table guidelines.
%
% For tables that exceed the width of the text column, use the adjustwidth environment as illustrated in the example table in text below.
%
% % % % % % % % % % % % % % % % % % % % % % % %
%
% -- EQUATIONS, MATH SYMBOLS, SUBSCRIPTS, AND SUPERSCRIPTS
%
% IMPORTANT
% Below are a few tips to help format your equations and other special characters according to our specifications. For more tips to help reduce the possibility of formatting errors during conversion, please see our LaTeX guidelines at http://journals.plos.org/plosone/s/latex
%
% For inline equations, please be sure to include all portions of an equation in the math environment.
%
% Do not include text that is not math in the math environment.
%
% Please add line breaks to long display equations when possible in order to fit size of the column.
%
% For inline equations, please do not include punctuation (commas, etc) within the math environment unless this is part of the equation.
%
% When adding superscript or subscripts outside of brackets/braces, please group using {}.
%
% Do not use \cal for caligraphic font.  Instead, use \mathcal{}
%
% % % % % % % % % % % % % % % % % % % % % % % %
%
% Please contact latex@plos.org with any questions.
%
% % % % % % % % % % % % % % % % % % % % % % % %

\documentclass[10pt,letterpaper]{article}
\usepackage[top=0.85in,left=2.75in,footskip=0.75in]{geometry}

% amsmath and amssymb packages, useful for mathematical formulas and symbols
\usepackage{amsmath,amssymb}

% Use adjustwidth environment to exceed column width (see example table in text)
\usepackage{changepage}

% Use Unicode characters when possible
\usepackage[utf8x]{inputenc}

% textcomp package and marvosym package for additional characters
\usepackage{textcomp,marvosym}

% cite package, to clean up citations in the main text. Do not remove.
% \usepackage{cite}

% Use nameref to cite supporting information files (see Supporting Information section for more info)
\usepackage{nameref,hyperref}

% line numbers
\usepackage[right]{lineno}

% ligatures disabled
\usepackage{microtype}
\DisableLigatures[f]{encoding = *, family = * }

% color can be used to apply background shading to table cells only
\usepackage[table]{xcolor}

% array package and thick rules for tables
\usepackage{array}

% create "+" rule type for thick vertical lines
\newcolumntype{+}{!{\vrule width 2pt}}

% create \thickcline for thick horizontal lines of variable length
\newlength\savedwidth
\newcommand\thickcline[1]{%
  \noalign{\global\savedwidth\arrayrulewidth\global\arrayrulewidth 2pt}%
  \cline{#1}%
  \noalign{\vskip\arrayrulewidth}%
  \noalign{\global\arrayrulewidth\savedwidth}%
}

% \thickhline command for thick horizontal lines that span the table
\newcommand\thickhline{\noalign{\global\savedwidth\arrayrulewidth\global\arrayrulewidth 2pt}%
\hline
\noalign{\global\arrayrulewidth\savedwidth}}


% Remove comment for double spacing
%\usepackage{setspace}
%\doublespacing

% Text layout
\raggedright
\setlength{\parindent}{0.5cm}
\textwidth 5.25in
\textheight 8.75in

% Bold the 'Figure #' in the caption and separate it from the title/caption with a period
% Captions will be left justified
\usepackage[aboveskip=1pt,labelfont=bf,labelsep=period,justification=raggedright,singlelinecheck=off]{caption}
\renewcommand{\figurename}{Fig}

% Use the PLoS provided BiBTeX style
% \bibliographystyle{plos2015}

% Remove brackets from numbering in List of References
\makeatletter
\renewcommand{\@biblabel}[1]{\quad#1.}
\makeatother



% Header and Footer with logo
\usepackage{lastpage,fancyhdr,graphicx}
\usepackage{epstopdf}
%\pagestyle{myheadings}
\pagestyle{fancy}
\fancyhf{}
%\setlength{\headheight}{27.023pt}
%\lhead{\includegraphics[width=2.0in]{PLOS-submission.eps}}
\rfoot{\thepage/\pageref{LastPage}}
\renewcommand{\headrulewidth}{0pt}
\renewcommand{\footrule}{\hrule height 2pt \vspace{2mm}}
\fancyheadoffset[L]{2.25in}
\fancyfootoffset[L]{2.25in}
\lfoot{\today}

%% Include all macros below

\newcommand{\lorem}{{\bf LOREM}}
\newcommand{\ipsum}{{\bf IPSUM}}






\usepackage{forarray}
\usepackage{xstring}
\newcommand{\getIndex}[2]{
  \ForEach{,}{\IfEq{#1}{\thislevelitem}{\number\thislevelcount\ExitForEach}{}}{#2}
}

\setcounter{secnumdepth}{0}

\newcommand{\getAff}[1]{
  \getIndex{#1}{1,2,3,4,5}
}

\providecommand{\tightlist}{%
  \setlength{\itemsep}{0pt}\setlength{\parskip}{0pt}}

\begin{document}
\vspace*{0.2in}

% Title must be 250 characters or less.
\begin{flushleft}
{\Large
\textbf\newline{Predicting seasonal influenza hospitalization using an ensemble super
learner: a simulation study} % Please use "sentence case" for title and headings (capitalize only the first word in a title (or heading), the first word in a subtitle (or subheading), and any proper nouns).
}
\newline
% Insert author names, affiliations and corresponding author email (do not include titles, positions, or degrees).
\\
Jason R. Gantenberg\textsuperscript{\getAff{1}, \getAff{2}}\textsuperscript{*},
Kevin W. McConeghy\textsuperscript{\getAff{2}, \getAff{3}},
Laura B. Balzer\textsuperscript{\getAff{4}},
Chanelle J. Howe\textsuperscript{},
\textsuperscript{\getAff{5}},
Andrew R. Zullo\textsuperscript{},
\textsuperscript{\getAff{1}, \getAff{2}}\\
\bigskip
\textbf{\getAff{1}}Department of Epidemiology, Brown University School of Public Health,
121 S. Main St., Providence, RI, 02912\\
\textbf{\getAff{2}}Department of Health Services, Policy and Practice, 121 S. Main St.,
Providence, RI, 02912\\
\textbf{\getAff{3}}Providence VA Medical Center, 830 Chalkstone Ave., Providence, RI, 02908\\
\textbf{\getAff{4}}Department of Biostatistics and Epidemiology, School of Public Health
and Health Sciences, University of Massachusetts Amherst, 427 Arnold
House, 715 N. Pleasant St., Amherst, MA 01003\\
\textbf{\getAff{5}}Department of Epidemiology, Center for Epidemiology and Environmental
Health, Brown University School of Public Health, 121 S. Main St.,
Providence, RI, 02912\\
\bigskip
* Corresponding author: jrgant@brown.edu\\
\end{flushleft}
% Please keep the abstract below 300 words
\section*{Abstract}
Lorem ipsum dolor sit amet, consectetur adipiscing elit. Curabitur eget
porta erat. Morbi consectetur est vel gravida pretium. Suspendisse ut
dui eu ante cursus gravida non sed sem. Nullam sapien tellus, commodo id
velit id, eleifend volutpat quam. Phasellus mauris velit, dapibus
finibus elementum vel, pulvinar non tellus. Nunc pellentesque pretium
diam, quis maximus dolor faucibus id. Nunc convallis sodales ante, ut
ullamcorper est egestas vitae. Nam sit amet enim ultrices, ultrices elit
pulvinar, volutpat risus.

% Please keep the Author Summary between 150 and 200 words
% Use first person. PLOS ONE authors please skip this step.
% Author Summary not valid for PLOS ONE submissions.
\section*{Author summary}
Lorem ipsum dolor sit amet, consectetur adipiscing elit. Curabitur eget
porta erat. Morbi consectetur est vel gravida pretium. Suspendisse ut
dui eu ante cursus gravida non sed sem. Nullam sapien tellus, commodo id
velit id, eleifend volutpat quam. Phasellus mauris velit, dapibus
finibus elementum vel, pulvinar non tellus. Nunc pellentesque pretium
diam, quis maximus dolor faucibus id. Nunc convallis sodales ante, ut
ullamcorper est egestas vitae. Nam sit amet enim ultrices, ultrices elit
pulvinar, volutpat risus.

\linenumbers

% Use "Eq" instead of "Equation" for equation citations.
\hypertarget{meta}{%
\section{Meta}\label{meta}}

\textbf{Target journal:} \emph{PLoS Computational Biology}

\noindent \textbf{Section:} Epidemiology and Clinical/Translational
Studies

\noindent \textbf{Potential editors:}

\begin{itemize}
\tightlist
\item
  Benjamin Althouse
\item
  Miles Davenport
\item
  Matthew Ferrari
\item
  Roger Kouyos
\item
  James Lloyd-Smith
\end{itemize}

\noindent \textbf{Potential reviewers:}

\begin{itemize}
\tightlist
\item
  Ryan J. Tibshirani (co-author on paper we use for curve simulation)
\item
  Logan C. Brooks (co-author on paper we use for curve simulation)
\item
  Roni Rosenfeld (co-author on paper we use for curve simulation)
\item
  Sherri Rose
\item
  David A. Osthus
\item
  Samrachana Adhikari
\end{itemize}

\hypertarget{introduction}{%
\section{Introduction}\label{introduction}}

Each year, seasonal influenza causes approximately XXXX hospitalizations
and XXXX deaths per year in the United States alone {[}cite{]}. Being
able to predict how influenza-related hospitalizataions will change over
time during any given influenza season can assist policymakers, public
health officials, and physicians allocate resources appropriately and
prepare more efficiently for changes in hospitalization rates.

While influenza forecasting is a still-maturing science {[}1{]},
researchers have made considerable progress over the past decade in
improving the quality of and capacity for forecasting influenza-like
illness (ILI) {[}cite{]}, thanks in part to the FluSight forecasting
competitions sponsored by the Centers for Disease Control and Prevention
(CDC) since the 2013--14 flu season {[}cite{]}. Many different types of
models have been used to generate forecasts, including statistical time
series models {[}1,2{]}, Bayesian regression {[}cite{]}, and agent-based
models {[}cite{]}, among others. However, ensemble methods have emerged
as perhaps the most promising approach to improving the accuracy and
stability of epidemic predictions {[}3,4{]}.

Ensembles combine predictions generated by a set of component models
{[}3,5--7{]}. In some cases, ensembles aggreggate component model
predictions by weighting better predictions more highly in the final
ensemble prediction {[}3,4{]}, though other weighting criteria can be
applied {[}4{]}. The rationale for using ensemble predictions rests in
their ability to borrow the strengths and discard the weaknesses of
various component models. This feature tends to lead not only to more
accurate predictions but to more stable ones that can be applied across
a range of scenarios {[}4{]}. The CDC's primary in-season ILI forecasts
are now based on an ensemble forecast generated by aggregating
predictions from a growing library of individual forecasts submitted by
research teams around the U.S. {[}{]}.

To date, most work has focused on ILI {[}1,2,8--10{]}, with considerably
less effort having been exerted so far on predicting influenza-related
hospitalization rates {[}11{]}. Because the dynamics of flu-related
hospitalizations might evolve differently over the course of an
influenza season---at the very least, lagging influenza incidence by a
week or two {[}citation needed{]}---and because hospitalization rates
are an independent signal of the severity of disease caused by
circulating flu strains, optimizing ensembles to predict hospitalization
rates can provide complementary information to ILI forecasts.

One ensemble machine learning method in particular, dubbed ``super
learner'' {[}12--14{]}, exhibits a number of desirable properties that
suggest it may be a powerful tool for predicting flu hospitalizations.
First, its developers have demonstrated that, asymptotically, the super
learner is an oracle estimator, performing as well as the best-fitting
component model and converging almost as quickly {[}13{]} {[}also will
want to read and cite the 2003 paper of van der Laan's{]}. Second, this
oracle property generally translates to finite samples {[}cite correct
Polley and van der Laan papers{]}. Finally, several packages have been
developed to implement the super learner algorithm {[}15,16{]},
providing researchers easy access to a relatively large library of
component models and a means to calculate cross-validated prediction
risks quite easily {[}16{]}.

In this study, we sought to train an ensemble learner on a distribution
of simulated influenza hospitalization curves to generate predictions
for three seasonal target parameters based on the CDC forecasting
competitions {[}17{]}: peak hospitalization rate, peak week of the
season, and cumulative hospitalization rate. We sought to compare the
performance of the ensemble learner to the best-performing component
model and a naive historical average prediction across the 30 weeks of a
flu season for each of these three prediction targets.

\hypertarget{methods}{%
\section{Methods}\label{methods}}

\hypertarget{results}{%
\section{Results}\label{results}}

\hypertarget{discussion}{%
\section{Discussion}\label{discussion}}

\hypertarget{software-and-code}{%
\section{Software and code}\label{software-and-code}}

All code is provided at \ldots{} {[}set up persistent DOI at Zenodo or
Open Science Framework and link to Github repo for FluHospPrediction
package{]}

\hypertarget{declarations}{%
\section{Declarations}\label{declarations}}

\hypertarget{acknowledgement}{%
\subsection{Acknowledgement}\label{acknowledgement}}

We thank Ashley Naimi and Nicholas Reich for their comments on earlier
versions of the analysis plan.

\hypertarget{funding-statement}{%
\subsection{Funding statement}\label{funding-statement}}

This work was funded by an unrestricted grant from Sanofi (PI: Andrew
Zullo). The funders did not assist in the statistical analysis nor did
they have a say in the final decision to submit the manuscript for
publication.

\hypertarget{competing-interests}{%
\subsection{Competing interests}\label{competing-interests}}

{[}solicit competing interests from co-authors{]}

\hypertarget{references}{%
\section*{References}\label{references}}
\addcontentsline{toc}{section}{References}

\bibliography{references}

\hypertarget{supporting-information}{%
\section*{Supporting information}\label{supporting-information}}
\addcontentsline{toc}{section}{Supporting information}

\hypertarget{refs}{}
\leavevmode\hypertarget{ref-Reich2019-uk}{}%
1. Reich NG, Brooks LC, Fox SJ, Kandula S, McGowan CJ, Moore E, et al. A
collaborative multiyear, multimodel assessment of seasonal influenza
forecasting in the united states. Proc Natl Acad Sci U S A. 2019;116:
3146--3154.
doi:\href{https://doi.org/10.1073/pnas.1812594116}{10.1073/pnas.1812594116}

\leavevmode\hypertarget{ref-Biggerstaff2018-ns}{}%
2. Biggerstaff M, Kniss K, Jernigan DB, Brammer L, Bresee J, Garg S, et
al. Systematic assessment of multiple routine and near Real-Time
indicators to classify the severity of influenza seasons and pandemics
in the united states, 2003-2004 through 2015-2016. Am J Epidemiol.
2018;187: 1040--1050.
doi:\href{https://doi.org/10.1093/aje/kwx334}{10.1093/aje/kwx334}

\leavevmode\hypertarget{ref-Reich2019-ca}{}%
3. Reich NG, McGowan CJ, Yamana TK, Tushar A, Ray EL, Osthus D, et al.
Accuracy of real-time multi-model ensemble forecasts for seasonal
influenza in the U.S. PLoS Comput Biol. 2019;15: e1007486.
doi:\href{https://doi.org/10.1371/journal.pcbi.1007486}{10.1371/journal.pcbi.1007486}

\leavevmode\hypertarget{ref-Ray2018-ef}{}%
4. Ray EL, Reich NG. Prediction of infectious disease epidemics via
weighted density ensembles. PLoS Comput Biol. 2018;14: e1005910.
doi:\href{https://doi.org/10.1371/journal.pcbi.1005910}{10.1371/journal.pcbi.1005910}

\leavevmode\hypertarget{ref-Hastie2009-ft}{}%
5. Hastie T, Tibshirani R, Friedman J. The elements of statistical
learning: Data mining, inference, and prediction. Springer, New York,
NY; 2009.
doi:\href{https://doi.org/10.1007/978-0-387-84858-7}{10.1007/978-0-387-84858-7}

\leavevmode\hypertarget{ref-Wolpert1992-pw}{}%
6. Wolpert DH. Stacked generalization. Neural Netw. 1992;5: 241--259.
doi:\href{https://doi.org/10.1016/S0893-6080(05)80023-1}{10.1016/S0893-6080(05)80023-1}

\leavevmode\hypertarget{ref-Breiman1996-ez}{}%
7. Breiman L. Stacked regressions. Mach Learn. 1996;24: 49--64.
doi:\href{https://doi.org/10.1007/BF00117832}{10.1007/BF00117832}

\leavevmode\hypertarget{ref-McGowan2019-ph}{}%
8. McGowan CJ, Biggerstaff M, Johansson M, Apfeldorf KM, Ben-Nun M,
Brooks L, et al. Collaborative efforts to forecast seasonal influenza in
the united states, 2015-2016. Sci Rep. nature.com; 2019;9: 683.
doi:\href{https://doi.org/10.1038/s41598-018-36361-9}{10.1038/s41598-018-36361-9}

\leavevmode\hypertarget{ref-Kandula2018-sq}{}%
9. Kandula S, Yamana T, Pei S, Yang W, Morita H, Shaman J. Evaluation of
mechanistic and statistical methods in forecasting influenza-like
illness. J R Soc Interface. 2018;15.
doi:\href{https://doi.org/10.1098/rsif.2018.0174}{10.1098/rsif.2018.0174}

\leavevmode\hypertarget{ref-Brooks2015-fl}{}%
10. Brooks LC, Farrow DC, Hyun S, Tibshirani RJ, Rosenfeld R. Flexible
modeling of epidemics with an empirical bayes framework. PLoS Comput
Biol. 2015;11: e1004382.
doi:\href{https://doi.org/10.1371/journal.pcbi.1004382}{10.1371/journal.pcbi.1004382}

\leavevmode\hypertarget{ref-Kandula2019-tg}{}%
11. Kandula S, Pei S, Shaman J. Improved forecasts of
influenza-associated hospitalization rates with google search trends. J
R Soc Interface. 2019;16: 20190080.
doi:\href{https://doi.org/10.1098/rsif.2019.0080}{10.1098/rsif.2019.0080}

\leavevmode\hypertarget{ref-Van_der_Laan2007-ml}{}%
12. Laan MJ van der, Polley EC, Hubbard AE. Super learner. Stat Appl
Genet Mol Biol. De Gruyter; 2007;6: Article25.
doi:\href{https://doi.org/10.2202/1544-6115.1309}{10.2202/1544-6115.1309}

\leavevmode\hypertarget{ref-Polley2010-cb}{}%
13. Polley EC, Laan MJ van der. Super learner in prediction. University
of California, Berkeley; 2010.

\leavevmode\hypertarget{ref-Polley2011-oz}{}%
14. Polley EC, Rose S, Laan MJ van der. Super learning. In: Laan MJ van
der, Rose S, editors. Targeted learning: Causal inference for
observational and experimental data. New York, NY: Springer New York;
2011. pp. 43--66.
doi:\href{https://doi.org/10.1007/978-1-4419-9782-1/_3}{10.1007/978-1-4419-9782-1\textbackslash{}\_3}

\leavevmode\hypertarget{ref-Polley2019-sl}{}%
15. Polley E, LeDell E, Kennedy C, van der Laan M. SuperLearner: Super
learner prediction {[}Internet{]}. 2019. Available:
\url{https://CRAN.R-project.org/package=SuperLearner}

\leavevmode\hypertarget{ref-Coyle2020-ze}{}%
16. Coyle JR, Hejazi NS, Malenica I, Sofrygin O. Sl3: Pipelines for
machine learning and super learning. 2020.
doi:\href{https://doi.org/10.5281/zenodo.1342293}{10.5281/zenodo.1342293}

\leavevmode\hypertarget{ref-Centers_for_Disease_Control_and_Prevention_undated-tx}{}%
17. Centers for Disease Control and Prevention. Epidemic prediction
initiative. \url{https://predict.cdc.gov/post/59973fe26f7559750d84a843};

\nolinenumbers


\end{document}

